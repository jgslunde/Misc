\documentclass[12p,a4paper]{article}
\usepackage[utf8x]{inputenc}
\usepackage[T1]{fontenc,url}
\usepackage{parskip}
\usepackage{lmodern}
\usepackage{microtype}
\usepackage{verbatim}
\usepackage{amsmath, amssymb}
\usepackage{tikz}
\usepackage{physics}
\usepackage{mathtools}
\usepackage{algorithm}
\usepackage{algpseudocode}
\usepackage{listings}
\usepackage{enumerate}
\usepackage{graphicx}
\usepackage{float}
\usepackage{epigraph}
\usepackage{hyperref}

\begin{document}
\section*{The xrange function: Increased performance in for loops}
The xrange function can replace the range function in any for loop with identical results and increased performance. There is no reason to use range instead of xrange in a for loop. The xrange function has a factor 2.3 higher performance, and eliminates the memory usage of the range function.

The range function creates a list and stores it in memory, which the for loop then iterates over.\\
The xrange function never creates a list, but simply gives the for loop each value as it comes along.


\end{document}
