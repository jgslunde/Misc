\documentclass[10pt,a4paper]{article}


%__Fonts and layout__
\usepackage[utf8x]{inputenc}     % Allows input of utf8 characters
\usepackage[T1]{fontenc,url}    % Helps correctly display all utf8 characters
\usepackage[english]{babel}     % Specifies language to be used by captions etc.
\usepackage{lmodern}    % Sets font-type to Latin Modern (which is considered pretty)
\usepackage{microtype}  % Improves text spacing and makes text prettier
\usepackage{parskip}    % Makes an empty line begin a new paragraph (inserts a vertical space)
\usepackage[margin=3cm]{geometry}   % Allows setting of margins(and other things)

\usepackage{verbatim}   % Verbatim enviroment, often used for code
\usepackage{listings}   % Listing enviroment, often used for code

%__math__
\usepackage{amsmath, amssymb}	% Improved math syntax and symbols
\usepackage{mathtools}  % Some matrix stuff and shit
\usepackage{siunitx}    % Allows use of SI-units, with correct spacing and text-type
\usepackage{tikz}   % Graph drawing

%__algorithms and programming__
\usepackage{algorithm}  % Allows writing of pretty algorithm. Perfect for desplaying code-syntax
\usepackage{algpseudocode}  % Similar to algorithm, just different layout
\usepackage{enumerate}  % For listing of stuff

%__graphs and pictures__
\usepackage{graphicx}   % Includegraphics
\usepackage{float}  % Allows the [H] option, to force graphics in place
\usepackage{color}  % Allows creation of colors

%__quotes and refrencing__
\usepackage{todonotes}  % Allowes adding todo notes to pdf
\usepackage{epigraph}   % Epigraph enviroment, for quotes
\usepackage{hyperref}   % Allows hyperreferences in pdf


\begin{document}

\newcommand{\half}{\frac{1}{2}}  %Defines a new command.
\renewcommand{\exp}{e^}


%__Making a first-page__
\title{This is the title}
\author{
	\begin{tabular}{rl}
		Author nr 1 & (\textit{username1})\\
		Author nr 2 & (\textit{username2})\\
	\end{tabular}}
\date{01.01.2000}
\maketitle



%__Epigraph__
\setlength{\epigraphwidth}{0.75\textwidth}
\renewcommand{\epigraphflush}{center}
\renewcommand{\beforeepigraphskip}{50pt}
\renewcommand{\afterepigraphskip}{100pt}
\renewcommand{\epigraphsize}{\normalsize}
\epigraph{This is a quote}
	{\textit{By this guy}}



%__Abstract__
\begin{abstract}
\noindent
This is an abstract
\end{abstract}


\newpage
\tableofcontents    % Only includes numbered sections
\newpage



%__Label and refrencing__
\section{Introduction}\label{sec:intro}
This is a reference to the Introduction\ref{sec:intro}


\begin{verbatim}
This is the verbatim enviroment
\end{verbatim}

\begin{lstlisting}
This is the listings enviroment
\end{lstlisting}


\section*{Math}
% Use the SI units package for correct display of units
\[
g = \SI{9.81}{m/s^2}
\]


\section*{Algorithm package}
\begin{algorithm}
\caption{Euler-Cromer} \label{alg:euler_cromer}
\begin{algorithmic}[1]
  \Procedure{EulerCromer}{$p, v, p\, v, m, n$}
    \For {$i \gets 0, \dots, n-1$}
        \State $acc \gets $ \textsc{Acceleration}$(p, v_{i}, m, i, n, dt)$
        \State $v \gets v_{i} + acc$
        \State $p \gets p_{i}$
    \EndFor
  \EndProcedure
\end{algorithmic}
\end{algorithm}


\section*{Include graphics}
\begin{figure}[H]
\centering	%Makes the graphic always stay in the center, no matter it's size.
\includegraphics[width=0.4\textwidth]{fig/sample_picture.png}
\includegraphics[width=0.3\textwidth]{fig/sample_picture.png}
\caption{This is a figure-caption}
\label{fig:figure1}
\end{figure}
This is a reference\ref{fig:figure1} to a figure.




\section*{Footnotes}
This is a text with a footnote\footnote{This is a footnote}



%__EndNotes
\newpage    % Begins a new page and leaves everything as is
\pagebreak  % Begins a new page but spreads the last page more evenly out where allowed
\vfill      % Fills the rest of the page with emptpy space
\vspace{1cm}    % Vertical spacing
\hspace{1cm}    % Horisontal spacing
\end{document}
